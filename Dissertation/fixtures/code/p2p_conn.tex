\begin{figure}[!h]
  \centering
  \begin{lstlisting}
      GNAT::Peer* peer = new GNAT::Peer();

      int successful = 0;
      successful = peer->openAsSessionHost();
      if (!successful) {
        // Log and abort
      }
  \end{lstlisting}
  \caption{Process of opening a peer up to connections from other peers}
  \label{code:peer_conn_host}
\end{figure}
\textbf{openAsSessionHost() returns:}
\begin{itemize}
\item A positive error code once the expected amout of clients have joined and the info about each client has been broadcast successfully to each client.

\item A negative error code will be returned if an unexpected error occurs. An example of this could be a timeout or failure to initialise Winsock on UDP port for game data or TCP port for client connections.
\end{itemize}


\begin{figure}[!h]
  \centering
  \begin{lstlisting}
      GNAT::Peer* peer = new GNAT::Peer();

      int successful = 0;
      // Session Host Details already configured
      successful = peer->connectToSessionHost();
      if (!successful) {
        // Log and abort
      }
  \end{lstlisting}
  \caption{Process for a peer connecting to peer session host}
  \label{code:peer_conn_join}
\end{figure}
\textbf{connectToSessionHost() returns:}
\begin{itemize}
\item A positive value once the client has successfully connected to the session host, received the join acknowledgement and successfully received client information about each other client.

\item A negative error code will be returned if an unexpected error occurs. An example of this could be a timeout or failure to initialise Winsock on UDP port for game data or TCP port for client connections. A negative value could also return if invalid data has been received when peer list was expected.
\end{itemize}
