\thispagestyle{plain}
\begin{center}
  \makeatletter
  \Large
  \textbf{\@title}

  \vspace*{0.4cm}
  \large
  Implementation and Analysis of Client-Hosted and Peer-to-Peer Networking Models in the Context of Games

  \vspace*{0.4cm}
  \textbf{\@author}

  \vspace*{0.9cm}
  \textbf{Abstract}
  \makeatother
\end{center}
A significant share of today's network traffic, consists of game networking. This is due to the emence amount of data that has to fundamentally be sent from client to client (or server) in order to keep a simulation synchronised between several participants. The fundamental principles of this mean that each client should be able to make a change (e.g. movement of their game character), and each other client that participates in the same instance of the simulation should be able to see this change on their local machine. There are several ways of achieving this.

The most common way of achieve a simulation that is synchronised across several users, is to introduce a trusted 3rd party entity that would essentially determine the ``Real'' state of the simulation. Each client participating in this instance, would submit their changes to this entity which would only implement the changes after checking for validity. The ``Real'' state would then be sent over to each client so that everyone can update their local simulations to it. Introducing a central entity like this provides a lot of advantages for everyone and is typically implemented by providing central servers that act as the 3rd party entity. The main advantages to this solution include fairness in latency and lag between all players and cheating prevention as all actions are approved before being accepted.

While this solution provides a lot of benefits to the playerbase of a game, it may not always be the best solution. The biggest problem that exists with this solution, is the cost that is associated with the deployment of these servers. Firstly, the cost of renting/running servers that are capable of handling the expected traffic and processing many instances of simulations at the same time could be very high. What makes this issue worse, is that if not enough of these servers are deployed, the experience of all players is compromised through large differences in player ping. This means that servers have to be deployed worldwide which involves further cost.

This downside can be overcome through avoiding the need for a central server through connecting to the other players directly, which introduces several compromises. The different ways of acomplishing a synchronised simulation, without a 3rd party entity, is what I will be investigating in this project through the implementation of 2 different ways of doing this; the Peer-to-Peer Model and the Client-Hosted model. I will be investigating the effectiveness of each solution across several scenarios and aim to come up with a guideline of which one should be used in given situations.
