\chapter{Introduction}
Throughout this project, I will be exploring the different methods of providing a synchronised, multiplayer gaming experience on two or more computers on a network. I will be covering the networking basics of how two game clients running on seperate machines can communicate with each other as well as exploring the relevant details of how and why the networking systems in games are designed the way they are.


I will explore the existing methods of writing an online multiplayer game from scratch and aim to provide an implementation of a networking library for online games to communicate without a central server. I also aim to provide analysis of efficieny of different methods of achieving this goal.

\section{Project Goals}
When developing a multiplayer game the most important aspect of the codebase is likely to be the networking functionality and speed. In most games designed for two or more players interacting with each other or the environment, the whole experience can be ruined by slow or inefficient networking implementation. The most common way for games publishers/developers to address this, is to rent out servers space optimised for fast network speeds and a large amount of processing power. This allows each player to connect to a central known entity that can be trusted to be fast enough to handle the load of processing each simulation step and broadcasting information quickly. This solution provides many benefits in terms of security, cheating prevention and fairness, however it is an expensive investment and risk for games developed on a smaller budget.

I will be investigating other options that exist for the smaller budget projects and how to make the gameplay as seamless as possible with the limitations of consumer hardware and network connections. While some aspects of this, such as high ping between clients due to a bad network connection or suboptimal routing are out of control of the implementation, there are many ways that the game netcode can be implemented to minimise latency and provide a synchronised experience for all clients.
\\
\\
Implementation goals:
\begin{itemize}
\item Implement a system that allows one client to act as a server for all others. Throughout this document, I will be refering to this as the \textbf{Client-Hosted} model.
\item Implement a system for each client to send their updates to each other client directly. Throughout this document, I will be refering to this as the \textbf{Peer-to-Peer} model.
\end{itemize}
Data gathering goals:
\begin{itemize}
\item TODO
\end{itemize}

\newpage

\section{Basic concepts and terminolory}
First of all, I will define some basic concepts and terminolory that will be used throughout this document.
\\
\\
\textbf{Client}: Within the context of this document, a client can be defined as a piece of software responsible to running the game (i.e. game client). A client can also be defined as a computer interacting with a server, however it is possible to run two different instances of a game client on a single machine. This can also be described as a networking endpoint.
\\
\\
\textbf{Online Multiplayer Game}: A video game can be defined as a simulation of a certain scenario that can be manipulated by the player of the game. When talking about online multiplayer games, it can be thought of as a simulation that runs on several clients connected by a network (e.g. LAN or the Internet) that is to be synchronised. When one player performs an action that effects the state of the simulation, this action should also be seen by all participants of this perticular simulation instance and therefore the simulation should remain in the same state across all participating clients.
\\
\\
\textbf{Ping}: In network connections, the ping between different clients refers to the shortest amount of time that is needed for one client to send information to another and receive a response from this client. One client sends a ``ICMP echo request'' to another networked client (e.g. a game server). The receiving client, then responds with an ``ICMP echo reply'' back to the original device. The time between sending the request and receiving the reply, is the ping between the two clients.
\\
\\
\textbf{Lag}: The grater the ping between two connected clients, the bigger the difference in the state of each clients' simulation once an action to be synchronised is performed. When a change is made by one client, this change should be seen by other clients participating in the same simulation and lag occurs when this change does not appear instentaneous to the user.
\\
\\
\textbf{Jitter}: The difference in frequency that the messages are sent from a sender and received by the receiver. If a sending client sends packets at a constant rate, they would ideally arrive at the receiver's client at the same rate. This is not always the case however and could lead to some unwanted results in certain applications such as VOIP.


\newpage
\section{Networking principles in games}
TODO: talk a bit about UDP, TCP/IP and routing...

\subsection{Communitaction Issues}
In the dissertation \mycite{macedonia1995network}, the author has identified and grouped the most prevelent issues that occur in internet communications. In real time applications such as online games, these issues can be very impactful on the experience of the players.

\subsubsection{Data Distribution}
Broadcast to each client

\subsubsection{Latency}
The paper \mycite{bettner20011500}, documents the architecture and implementation of networking approaches in the RTS (Real Time Startegy) games ``Age of Empires'' 1 and 2. The authors talk about an experiment that has been performed with the game netcode where players were interviewed on whether they felt that their game inputs felt responsive at differnet latencies between the players. They have found that when playing with a latency of 250ms or less, the latency was not noticible at all. Latencies between 250ms and 500ms were ``very playable''. Anything more than 500ms would start becoming noticible however. They have also found that when playing with a noticible latency, the players have naturally developed a ``mental expectation of the lag'' between the inputs and an action happening. After developing this ``game pace'', they would enjoy playing the game at a consistent slower response than altering between slow and fast letency.

What can be seen in this example, is that while making the latency between inputs and actions, does make the game more responsive and fun, noticible jitter when receiving messages can also impact the player's experiance in significant ways.


\subsubsection{Reliability}
UPD can lose packets........ The article \mycite{lincroft1999internet} documents the issues that the developement team encountered when developing the netcode for the 1997 game ``X-Wing vs. TIE fighter''. The author has experimented with TCP connections in game data transmission and found that ``TCP refuses to deliver any of the other packets in the stream while it waits for the next "in order" packet. This is why we would see latencies in the 5-second range.'' and ``if a packet is having a tough time getting to its destination, TCP will actually stop re-sending it! The theory is that if packets are being dropped that it's due to congestion.''. The features that have been implemented into TCP to make it reliable, end up negatively effecting real-time application data traffic and due to the volume of data to be transmitted, should not be used for time sensitive data.

\subsubsection{Bandwidth}
Consumer hardware and networking slow (esp upload)

\subsection{Inconsistant ping between players}
There are many different possible reasons for the lag to the server to vary widely from player to player. Firstly, it is possible that there are not enough servers throughout the world or that they are not spread out evenly enough across all regions. Players who live in geographically more remote locations, are likely to experience higher ping to servers compared to those that live in more densly populated cities due to where the servers are likely to be located. Secondly, there is no way of guaranteeing if the clients are going to be using a wired or a wireless connection to connect to the server. Wireless connections can be much slower and are more likely to introduce other potential problems such as packet loss.

Varying ping between players could lead to a problem of a poor experience for a client with a high ping to the server, as they will receive the updates from the server later than every other player and therefore be at a disadvantage if the game requires real time reactions. Unfortunately under some implementations, this also results in a poor experience for the other players too, who despite having resonable ping to the server, can be shot from behind cover by a laggy player who fired a shot before the cover was reached by them on their version of the game state which is delayed compare to others.


\subsubsection{Possible solutions to the variable ping problem}
Netcode developers of the most popular games, have tried many different solutions to fix or at least mitigate the issue of widely differing ping to the game server between players. One example of a solution here could be ``region locking''. This is the idea that only players with equally low ping, can connect to the same game instance on a server. This could be done by providing game servers spread out across as many geographical regions as possible and only allow players to connect to their local one. This presents two main issues however. Firstly, this prevents players in different geographical regions from playing together and therefore serperates the community. Also, the issue of players in remote areas playing together and having a suboptimal experience due to the server lag, has not been addressed. The developers of Battlefield 1 have implemented an interesting solution to this problem and I will discuss this further on.


\newpage
\section{Networking Model Options}
TODO; dev have choice. all have positives and negatives.


\subsection{The Centralised Server Model}
Most AAA online multiplayer games that are played today, make use of the ``centralised server model'' for synchronising the simulation state between several clients participating in the same simulation. This means that in an example of a First Person Shooter (FPS), if one player presses the ``jump'' key, their character will jump on their screen. This input information would then also be sent over to the game server. The server would then send the information that this player has jumped, to all other clients. There is a potential problem here however. Given that the ping between the server and client A is \(\alpha\) and between the server and client B is \(\beta\), the time between client A pressing an input and client B being notified of this input can not be less than $\alpha+\beta$ and due to the limitations of physics $\alpha>0$ and $\beta>0$. This means that at any given time the lag experienced between clients A and B will be more than $\alpha+\beta$ when aspects such as server tickrate or network congestion are factored in too.

\begin{figure}[!h]
  \begin{tabular}{ c p{0.94\textwidth} }
    \faCheckCircle & Hardware is likely to be powerful enough to handle the stress of many simulations running simultaneously. \\
    \faCheckCircle & High bandwidth connection is likely to be used. Minimises the chance of high latency and network issues. \\
    \faCheckCircle & Ping to the server is likely to be similar for all players making the game more fair.  \\
    \faCheckCircle & No client can see the address of any other client. \\
    \faCheckCircle & The developer has a lot of control over what is allowed within the game. This allows for anti-cheating systems that are hard to bypass. \\
    \  & \  \\
    \faTimesCircle & Expensive to rent out, or buy and maintain, server space. Letting players rent out servers makes the game more expensive for them. \\
    \faTimesCircle & Many servers have to be spread out evenly throughout the world to allow for low latency connections.  \\
    \faTimesCircle & People living in remote locations may not have low latency access to official servers.
  \end{tabular}
  \caption{The attributes of the central server model}
  \label{fig:cs_attributes}
\end{figure}



\subsection{Client Hosted Model}
The client hosted model is an example of implementing a networking infrastructure without the need for expensive server rental. One of the main advantages of this solution is the idea that one codebase can be written to work as a client hosted model and this codebase can then be adjusted slightly to work on a central server as well if there ever is a need for this in the future. The basic idea behind this implementation method, is that one of the clients, would act as a server for all other participants alongside also being one of the participants. One of the biggest problems with an implementation like this, is that the player that is hosting the game will have a connection to the server with the ping and latency of 0 which will give them an objective advantage in scenarios where quick reaction times are needed. Another potential concern with an implementation like this, is that the player hosting the game, is likely to have consumer grade, lower-end hardware and network connection. It is also possible that a WiFi connection would be used increasing the chance of packet loss and high latency further. There is also a security concern with any networking application that, in order to function, has to know the IP address of every player that is connected. This information can be easily used to determine the country or even the city that the player is connecting from. It would also be possible to find out information about their hardware or network connection through how frequently a packet arrives from this player. A difficulty can also arrise if the host player loses connection to the other players as either the game has to finish or a ``host migration'' would have to take place which is a difficult problem.

\begin{figure}[!h]
  \begin{tabular}{ c p{0.94\textwidth} }
    \faCheckCircle & There are no additional costs to the publisher/developer when releasing the game. \\
    \faCheckCircle & Players in remote locations can play together with low latency. \\
    \faCheckCircle & The codebase is relatively easily transferable to a central server model if the need arises. \\
    \  & \  \\
    \faTimesCircle & The host player has negligible ping to the server. This could be a large advantage.  \\
    \faTimesCircle & The host player is likely to use a consumer grade connection increasing the risk of packet loss and high latency \\
    \faTimesCircle & The host player could be using WiFi to host the game which could significantly increase the chance of packet loss. \\
    \faTimesCircle & The host's hardware may not be powerful enough to calculate each simulation step within an acceptable tick rate. \\
    \faTimesCircle & If the host player is disconnected mid-game, a host migration will have to take place pausing the game for a few seconds or causing the game to finish unexpectedly. \\
    \faTimesCircle & The host player can see the IP address of each other player that they are playing with. \\
    \faTimesCircle & Cheating could be easy if the host player sends malicious packets to the clients pretending to be the game server.

  \end{tabular}
  \caption{The attributes of the client hosted model}
  \label{fig:ch_attributes}
\end{figure}



\subsection{Peer to Peer Model}
The idea behind the peer to peer model, is that if two clients can communicate directly, the latency in the connection between them, could be reduced if where wasn't a server that all messages have to go through. The theory here is that the latency beteen players can be as low as theoretically possible which should result in more responsive gameplay. The use of the peer to peer model is often reserved for applications where only a few clients need to communicate with one another and therefore is a common choice for fighting games and RTS (Real Time Strategy) Games.

\begin{figure}[!h]
  \begin{tabular}{ c p{0.94\textwidth} }
    \faCheckCircle & There are no additional costs to the publisher/developer when releasing the game. \\
    \faCheckCircle & Players in remote locations can play together with low latency. \\
    \faCheckCircle & There is no concept of host advantage, like what is present with the client hosted model. \\
    \faCheckCircle & Host migrations are not a large problem since everyone has the full simulation state. \\
    \  & \  \\
    \faMinusCircle & Each client communicates directly with other clients making their connection as efficient as possible in theory. \\
    \faMinusCircle & Even though a central ``authority'' is not needed, one of the peers often has to act as a session host to handle invitations and handshakes. \\
    \faMinusCircle & Every client runs it's own simulation and is tasked with keeping it updated with everyone else's. \\
    \  & \  \\
    \faTimesCircle & The lack of a central authority (such as a game server) makes cheat prevention difficult. \\
    \faTimesCircle & Each player in an instance, can see the IP address of every other player they are playing with. \\
    \faTimesCircle & Interacting with two different peers with different latencies, will feel inconsistant for the player.  \\
    \faTimesCircle & Very high bandwidth usage compared to other models. \\
    \faTimesCircle & The amount of update messages that need to be sent, increases as the number of players grows. \\
    \faTimesCircle & A player with a poor internet connection or underpowered hardware, will make the game feel less responsive to other players. \\
   \end{tabular}
  \caption{The attributes of the peer to peer model}
    \label{fig:p2p_attributes}
\end{figure}

