\chapter{Background}

\section{Networking principles in games}
TODO: talk a bit about UDP, TCP/IP and routing...

\subsection{Communitaction Issues}
In the dissertation \mycite{macedonia1995network}, the author has identified and grouped the most prevelent issues that occur in internet communications. In real time applications such as online games, these issues can be very impactful on the experience of the players.

\subsubsection{Data Distribution}
Broadcast to each client

\subsubsection{Latency}
The paper \mycite{bettner20011500}, documents the architecture and implementation of networking approaches in the RTS (Real Time Startegy) games ``Age of Empires'' 1 and 2. The authors talk about an experiment that has been performed with the game netcode where players were interviewed on whether they felt that their game inputs felt responsive at differnet latencies between the players. They have found that when playing with a latency of 250ms or less, the latency was not noticible at all. Latencies between 250ms and 500ms were ``very playable''. Anything more than 500ms would start becoming noticible however. They have also found that when playing with a noticible latency, the players have naturally developed a ``mental expectation of the lag'' between the inputs and an action happening. After developing this ``game pace'', they would enjoy playing the game at a consistent slower response than altering between slow and fast letency.

What can be seen in this example, is that while making the latency between inputs and actions, does make the game more responsive and fun, noticible jitter when receiving messages can also impact the player's experience in significant ways.


\subsubsection{Reliability}
UPD can lose packets........ The article \mycite{lincroft1999internet} documents the issues that the developement team encountered when developing the netcode for the 1997 game ``X-Wing vs. TIE fighter''. The author has experimented with TCP connections in game data transmission and found that ``TCP refuses to deliver any of the other packets in the stream while it waits for the next "in order" packet. This is why we would see latencies in the 5-second range.'' and ``if a packet is having a tough time getting to its destination, TCP will actually stop re-sending it! The theory is that if packets are being dropped that it's due to congestion.''. The features that have been implemented into TCP to make it reliable, end up negatively effecting real-time application data traffic and due to the volume of data to be transmitted, should not be used for time sensitive data.

\subsubsection{Bandwidth}
Consumer hardware and networking slow (esp upload)
Talk about how MTU Works

\subsection{Inconsistant ping between players}
There are many different possible reasons for the lag to the server to vary widely from player to player. Firstly, it is possible that there are not enough servers throughout the world or that they are not spread out evenly enough across all regions. Players who live in geographically more remote locations, are likely to experience higher ping to servers compared to those that live in more densly populated cities due to where the servers are likely to be located. Secondly, there is no way of guaranteeing if the clients are going to be using a wired or a wireless connection to connect to the server. Wireless connections can be much slower and are more likely to introduce other potential problems such as packet loss.

Varying ping between players could lead to a problem of a poor experience for a client with a high ping to the server, as they will receive the updates from the server later than every other player and therefore be at a disadvantage if the game requires real time reactions. Unfortunately under some implementations, this also results in a poor experience for the other players too, who despite having resonable ping to the server, can be shot from behind cover by a laggy player who fired a shot before the cover was reached by them on their version of the game state which is delayed compare to others.


\subsubsection{Possible solutions to the variable ping problem}
Netcode developers of the most popular games, have tried many different solutions to fix or at least mitigate the issue of widely differing ping to the game server between players. One example of a solution here could be ``region locking''. This is the idea that only players with equally low ping, can connect to the same game instance on a server. This could be done by providing game servers spread out across as many geographical regions as possible and only allow players to connect to their local one. This presents two main issues however. Firstly, this prevents players in different geographical regions from playing together and therefore serperates the community. Also, the issue of players in remote areas playing together and having a suboptimal experience due to the server lag, has not been addressed. The developers of Battlefield 1 have implemented an interesting solution to this problem and I will discuss this further on in Section \ref{sec:bf1_ping}.


\newpage
\section{Networking Model Options}
TODO; dev have choice. all have positives and negatives.


\subsection{The Centralised Server Model}
Most AAA online multiplayer games that are played today, make use of the ``centralised server model'' for synchronising the simulation state between several clients participating in the same simulation. This means that in an example of a First Person Shooter (FPS), if one player presses the ``jump'' key, their character will jump on their screen. This input information would then also be sent over to the game server. The server would then send the information that this player has jumped, to all other clients. There is a potential problem here however. Given that the ping between the server and client A is \(\alpha\) and between the server and client B is \(\beta\), the time between client A pressing an input and client B being notified of this input can not be less than $\alpha+\beta$ and due to the limitations of physics $\alpha>0$ and $\beta>0$. This means that at any given time the lag experienced between clients A and B will be more than $\alpha+\beta$ when aspects such as server tickrate or network congestion are factored in too.

\begin{figure}[h!]
  \begin{tabular}{ c p{0.94\textwidth} }
    \faCheckCircle & Hardware is likely to be powerful enough to handle the stress of many simulations running simultaneously. \\
    \faCheckCircle & High bandwidth connection is likely to be used. Minimises the chance of high latency and network issues. \\
    \faCheckCircle & Ping to the server is likely to be similar for all players making the game more fair.  \\
    \faCheckCircle & No client can see the address of any other client. \\
    \faCheckCircle & The developer has a lot of control over what is allowed within the game. This allows for anti-cheating systems that are hard to bypass. \\
    \faTimesCircle & Expensive to rent out or buy and maintain server space. Letting players rent out servers makes the game more expensive for them. \\
    \faTimesCircle & Many servers have to be spread out evenly throughout the world to allow for low latency connections.  \\
    \faTimesCircle & People living in remote locations may not have low latency access to official servers.
  \end{tabular}
  \caption{The attributes of the central server model}
\end{figure}



\subsection{Client Hosted Model}
The client hosted model is an example of implementing a networking infrastructure without the need for expensive server rental. One of the main advantages of this solution is the idea that one codebase can be written to work as a client hosted model and this codebase can then be adjusted slightly to work on a central server as well if there ever is a need for this in the future. The basic idea behind this implementation method, is that one of the clients, would act as a server for all other participants alongside also being one of the participants. One of the biggest problems with an implementation like this, is that the player that is hosting the game will have a connection to the server with the ping and latency of 0 which will give them an objective advantage in scenarios where quick reaction times are needed. Another potential concern with an implementation like this, is that the player hosting the game, is likely to have consumer grade, lower-end hardware and network connection. It is also possible that a WiFi connection would be used increasing the chance of packet loss and high latency further. There is also a security concern with any networking application that, in order to function, has to know the IP address of every player that is connected. This information can be easily used to determine the country or even the city that the player is connecting from. It would also be possible to find out information about their hardware or network connection through how frequently a packet arrives from this player. A difficulty can also arrise if the host player loses connection to the other players as either the game has to finish or a ``host migration'' would have to take place which is a difficult problem.

\begin{figure}[!h]
  \begin{tabular}{ c p{0.94\textwidth} }
    \faCheckCircle & There are no additional costs to the publisher/developer when releasing the game. \\
    \faCheckCircle & Players in remote locations can play together with low latency. \\
    \faCheckCircle & The codebase is relatively easily transferable to a central server model if the need arises. \\
    \  & \  \\
    \faTimesCircle & The host player has neglegible ping to the server. This could be a large advantage.  \\
    \faTimesCircle & The host player is likely to use a consumer grade connection increasing the risk of packet loss and high latency \\
    \faTimesCircle & The host player could be using WiFi to host the game which could significantly increase the chance of packet loss. \\
    \faTimesCircle & The host's hardware may not be powerful enough to calculate each simulation step within an accptable tick rate. \\
    \faTimesCircle & If the host player is disconnected mid-game, a host migration will have to take place pausing the game for a few seconds or causing the game to finish unexpectadly. \\
    \faTimesCircle & The host player can see the IP address of each other player that they are playing with. \\
    \faTimesCircle & Cheating could be easy if the host player sends malicious packets to the clients pretending to be the game server.

  \end{tabular}
  \caption{The attributes of the client hosted model}
  \label{fig:ch_attributes}
\end{figure}



\subsection{Peer to Peer Model}
The idea behind the peer to peer model, is that if two clients can communicate directly, the latency in the connection between them, could be reduced if where wasn't a server that all messages have to go through. The theory here is that the latency beteen players can be as low as theoretically possible which should result in more responsive gameplay. The use of the peer to peer model is often reserved for applications where only a few clients need to communicate with one another and therefore is a common choice for fighting games and RTS (Real Time Strategy) Games.

For the peer to peer model to work, each client (peer), needs to know the IP address of every other client. This presents a potential security risk as any player in the same instance of the simulation, can use the IP address to determine what country, and sometimes city, each client is connecting from. Another potential problem arrises with how the information about each client is distributed to each client. One potential idea of how this could be done, is to have a central server that manages sessions and matchmaking. Clients that are looking for a session with an open spot, would send a join request message to this known server address. This server would match players up based on factors such as player skill. Once enough players have requested to join to start the simulation, the server would broadcast the IP and Port of each client to each other client. From that point, each client has all the information that it needs to start the simulation execution.

If the goal of using the peer to peer model is to avoid the need for server rental or maintenance, this process could also be done with one of the players hosting the game and having the other players know the address of this hosting client. With this setup, the scenario presented above, would work too.

Since each player is communicating with every other client, there is no central ``real state'' of the simulation and due to latency in connections, the simmulation from each player's perspective could differ slightly. This also means that each client is responsible for maintaining their own state of the simulation. Another conciquence of contacting other clients directly, is that in a randomly selected sample of players in a game that is available worldwide, the latency between each player is likely to vary largely. This can often result in having a slower response when interacting with one player than another who could have a lower latency to you. This could make the game feel unresponsive at times seemingly out of nowhere during gameplay.

The paper \mycite{Carter2009} provides an example of a game called ``Panzer Battalion'', where ``each peer computes its own game version, no player is having disadvantages because of high network lags''. This is an example of a game that uses the peer to peer model to allow for players to play together. The authors also explain despite removing a central point of failure, an implementation like this has led to ``contradicting game states'' which could potentially ruin the game experience.


\begin{figure}[h!]
  \begin{tabular}{ c p{0.5\textwidth} }
    \faCheckCircle & Test good \\
    \faMinusCircle & Test attribute \\
    \faTimesCircle & Test bad
  \end{tabular}
  \caption{The attributes of the peer to peer model}
\end{figure}



\chapter{Related work}
Throughout this chapter, I will be investigating how networking is implemented in popular modern AAA games. I will also implement a simple online, multiplayer demo game using different ways of implementing networking in the Unity game engine. The experience of using the options that exist in Unity for network game developement, will be similar to what I will aim to implement at a much lower level with my networking template.

\section{Networking implementations in games}

\subsection{Variable ping system in Battlefield 1} \label{sec:bf1_ping}
An interesting approach to the issue of variable ping in a game has been implemented by DICE in the game Battelfield 1. Given 2 players; player A with a low ping to the server and player B with a ping of <150ms to the server.

When player B fires at a moving player A, player B's client will perform the check concluding that player A has been hit and this information is sent to the server. The server will then perform it's own checks and if the server agrees that this hit is possible, then it sends the hit confirmation to player B and damage information to player A. This approach is called Clientside-Server Authoritative as while the hit registration is calculated on clientside, the server must still confirm that this is valid.

Concidering another scenario, suppose that player A still has a low ping to the server but player B, now has the ping of >150ms. An icon will appear on player B's UI showing an ``aim-lead'' indicator. Now when the shot is fired in the same scenario, the hit will not register anymore as the hit registration has switched from Clientside-Server Authoritative to Fully-Server Authoritative, meaning that the check is performed only once the shot information is received by the server.

Whilest this implementation makes the game feel less responsive for players with high ping, it provides a lot more fairness for everyone else and allow for players with different pings play in a more fair way.


\subsection{Networking in Apex Legends}
Apex Legends is a ``Battle Royale'' game by Respawn Entertainment. Due to the genre of this game, the implementation of networking has been implemented in an interesting way. The premise of the game consists of a starting amount of 60 players, all competing to be the last one alive at the end. An interesting aspect of this, is that while at the start of a match, the server might have to work quite hard to effectively synchronise the simulation for 60 different clients, as more and more players die off and less are left, less information needs to be sent to each of the clients, freeing up some server performance.

Since transferring all the information about 60 different players would most likely exceed the average MTU of a UDP packet, with each update, the server sends 1 smaller packet per each player to each player. This means that when 60 players are in the game, 60 packets are sent to 60 different clients from the same server. This is shown in the Netcode analysis in \mycite{bns2019apex}.

An interesting outcome of this, is quite noticible network lag that slowly goes away, as less and less players remain in the game. It is also likely due to this, that Respawn felt the need to implement a ``Client Authoritative'' aiming model. Meaning that the server will favor what the client sees over it's own view of the simulation (i.e. If the client claims that a shot has hit an enemy, the server is likely to agree). The choise for this implementation could have been made to fix two potential issues; reducing the load on the server by reducing the need to perform extencive checks for each shot (this could be significant if a lot of players are playing), making the game feel more responsive on slower client hardware that may need more time to process up to 60 packets that arrive from the server at each update tick.

\subsection{Destiny 2's uniquely complicated netcode}
In 2015, a Bungie developer gave an interseting talk at GDC (Game Developer's Conference): \mycite{truman2015destiny2}. The game Destiny 2, has decided to use the peer to peer model to create an ``always online'' world that players could join and leave freely. This resulted in a netcode that was refered to by \author{truman2015destiny2} as having a  ``uniquely complicated network topology''. In a previous Bungie game with multiplayer networking, Halo Reach,  a mixture of a peer to peer model and client hosted model has been used in multiplayer matches. Each player would send information such as weapon fire or movement to all other players but there would also be a physics server running on one of the clients in the simulation. The matches were designed to be relatively short and players where incentivised to remain in the game until the end of each match. This meant that if the player running the physics server were to leave, the gameplay flow would have to be interrupted for all other players as the host migration system allocated a new player to run this server. Fortunately, due to the nature of the game, this happened rarely as players were incentivised to remain till the end of the match. In Destiny 2 however, there are many zones (refered to in the talk as ``bubbles'') that a players would often join and leave as they move around the environment. This made host migrations much more common and would happen roughly ``every 160 seconds''. This has prompted Bungie to abandon the client hosted physics server aspect of Destiny 2's netcode and move towards renting cloud server space to host the physics engine. In ``private bubbles'', however, it became inefficient to communicate with a central server for game physics calls so the old system was used there. In summary, Destiny 2 uses peer to peer in order to send packets directly to other players to make the gameplay feel more responsive than sending messages to the server and then a client but has fixed some issues that the game could have if it was fully peer to peer by implementing elements that use client hosted and central server models too.

\begin{figure}[!h]
  \centering
  \includegraphics[width=0.5\textwidth]{Destiny2_networking}
  \caption{Example of how netcode is designed in Destiny 2. Image from the GDC talk: \mycite{truman2015destiny2}}
  \label{fig:destiny2netcode}
\end{figure}

\subsection{Issues with the peer to peer system}
The largest issues plaguing the games implemented with the peer to peer system all can be traced back to inconsistancies of state between the peers. This is inevetable when a delay is present in communication between parties. The paper \mycite{Carter2009} discusses some possible solutions that aim to mitigate or eliminate the idea of desynchronised game state. In general, during network communication, it can be sometimes be assumed that the connection is reliable however we can not assume that messages will arrive in the same order as they are sent.

The first idea that is suggested is called ``dead reckoning''. The paper \mycite{smed2002review} discusses this topic in detail. This is the idea of estimating what value is likely to be received from the server in the next packet by evaluating the previous values. In an example of an entity moving through a virtual environment at a constant speed in a constant direction and given that the position of this entity is being broadcasted from another entity, the game client can predict that if the previous packets have updated the position of this entity by the same amount every time, it can calculate the next value that will be received before the packet even arrives. This information can be used to interpolate the position of this entity between the packet that has just arrived and what poition it is expected to be in after the next packet arrives, thus giving the client a more smooth simulation. This also makes the smoothness of the simulation less reliant on network quality, though with low quality networking, the issue of ``rubber banding''\footnote{Rubber Banding is when a client sees an object in a simulation in a certain location but after a update of where this object should be arrives from an authority (server), this position is instantly updated to this new, correct position. From the client's point of view, this looks like the entity has teleported to the new location.} can become prevelent.

Another idea suggested is called ``local lag'' which involves the idea of puting incoming packets into a queue before they are used in the game. Using this method, if packets arrive out of order, they can be put in-order in the local buffer. In some game implementations, two actions performed in different order could result in a different outcome (for example rotating a character and moving them forwards). A potential downside that can arrise from using this method however, involves the fact that additional delay is incorporated to the process of sending information. This makes use of a system like this in games that require timely reactions potentially degrading for the players' experience. Local lag can be implemented in conjunction with dead reckoning for potentially better results.

Finally, an option called ``time warp'' can be utilised for synchronising game state. This technique is often used and well known in the field of Parallel and distributed Discrete Event Simulation (PDES). The paper \mycite{Jefferson1985} explains this topic in detail and ``Time Warm Mechanism'' is well explained. This mechanism listens and applies updates received from other peers when they arrive however, periodic ``snapshots'' of the entire game state are saved. If an inconsistancy is observed (for example if a straggler message is received from another peer), a ``rollback'' is applied to reset the simulation to a previous state, before the inconsistancy was observed. The system would also have to assume that any messages sent before the rollback was performed, were done so under inacurate assumptions. If this is the case, the peer can send a ``null-message'' which would inform the other peers of the inconsistancy and lead them to performing their own ``rollbacks''. This system can often cause ``network floods'' of ``null-messages''.

The system of ``local lag'' assumes that inconsistencies will always occur whereas ``time warp'' can often be concidered to be ``optimistic approaches'' because they allow each client to calculate their simulation state assuming no inconsistancies. Both implementations have uses in different scenarios.

\section{Other networking developement tools}
In this section, I will identify and investigate some tools that are available to developers for networked game developement as well as potential new technologies that could make a large imact on multiplayer online games.


\subsection{Google's Stadia and future networking options}
At the Game Developers Conference in March 2019, Google has announced a new Game Streaming platform. The talk outlined the plans that would allow developers to develop games for their Linux based platform that would run the game code on Google servers. Users of this service, would be able to play games by sending their inputs to the google servers. All game logic processing would then take place on the Google servers and the audio and video would be returned to the player. This service could potentially allow players to play games on devices with very little processing power, with settings that would normally require much more power.

A potential unexpected advantage of this system, could allow for networking possibilities that would be difficult to make work with current technologies. If a multiplayer game was developed for this system, it could allow for large scale virtual worlds with many players in the same instance as only one simulation would have to be processed for all the players. Each player's perspective would then be sent to them, this wouldn't be much different from how this system would function in a single player scenario.

The largest potential issue with a system like this could be the latency between the player sending an input to the server and then seeing this input reflected in the output. If this latency can be made neglegible through high speed internet connections and high processing power servers, it is possible that a system like this could work. If this latency can be minimised however, this system could prove to be a model that provides the best networking performance for large scale multiplayer experiences.


\subsection{Networking options in Unity}
A game engine has been described as a core of controlling games by providing the main framework and common functions in the paper \mycite{xie2012research}. Unity is just one example of many different game engines that are avaliable to a developer, which all aim for the same unified goal but try to achieve it in different ways. Unity is a popular choice for projects ranging from small indie titles to large AAA productions and it is known for being relatively easy to use and prototype with. To investigate the options that are available in the market for online game developement, I have decided to investigate how networking functionality can be implemented in this game engine.

I have implemented a simple 2D game which allows the player fire a projectile, jump and move their character left and right. The goal with this project was to go through the process of transforming a simple, single player game to a simple multiplayer game through the use of different options available in Unity.

\subsubsection{UNET}
The first method of networking that I have decided to investigate was Unity's own UNET library. UNET uses the client hosted model that requires one client to act as a game server allowing others to connect to it. The joining clients need to know the IP address of the host in order to connect to them and the host can either join the game themselves or act purely as a server. The figures \ref{fig:unet_ui}, \ref{fig:main} and \ref{fig:in_game}, show screenshots of the default UI that is provided by the library as well as my game showing the main menu, client and server perspectives respectively. The tile-based pixel art used in the game has been found on the Unity Asset Store under the name ``Sunny Land Forest'' and has been designed by Ansimuz.

Initially, the implementation of networking features proved to be very simple. A networking manager had to be added to the main scene and the networked scene had to be chosen. Spawn points, and the order that they should be used in, was easy to configure and the UI that allowed the user to connect and play was a simple addition too. Once two different player characters were spwaned in, there were some changes that had to be made to the movement scripts that involved specifying different behaviour whether the given character ``belongs'' to this player, or the other player. Checking if the character belonged to the player was as simple as comparing one boolean. This change can be seen in the change of colour of the character to blue if it is not the player controlled character in figure \ref{fig:in_game}.

When testing this version of the game, I have found that the movement felt just as smooth inside the multiplayer game session as it did in the single player game. When both the host and the client are playing on localhost (this should in theory provide the lowest latency possible for communications), player 1's movement, would be roughly translated to player 2's perspective correctly, however instead of the movement appearing as smooth as it does for the local player's movement, the enemy character seems to teleport around with every frame appearing in a different location in a radius around where they ``should'' be at that time. This could be due to the UNET implementation of dead reckoning that does not predict the upcoming positions very well. This teleportation problem occurs for both the host and the client. Another issue appears when a fast moving projectile is fired. For the host, all projectiles fly just as they should, in a straight line and colliding with a collider. On the client's screen however, it appears as if the tickrate of the server cannot keep up with the speed of this projectile. Often there are only 1 or 2 frames when the projectile sprite can be seen flashing on the screen in different positions before not being shown again. This appears to happen for both player's projectiles on the client's game. There is another issue with this implementation of my game on top of this; the desynchronisation of the two players happens very often. After player 2 stops moving, on player 1's screen, player 2's character will continue moving as if a force was being applied in a certain direction on the character collider. This very often results in the two simulations showing completely different states to eachother making it hard to know where the other player really is.

All these issues could be fixed by implementing techniques designed for minimising de-sync in peer to peer games and/or sending and reading custom messages between the clients that are designed to work with the game logic, however the default implementation of the official library that is designed to synchronise prefab positions and colliders, appear to work too slowly to be truely useful for quick moving objects, even in the most optimal conditions.

\begin{figure}[p]
  \centering
  \subfloat[Main Menu UI]{
    \includegraphics[width=0.3\linewidth]{UNET/Main_Menu_UI}
  }
  \qquad
  \subfloat[Host UI]{
    \includegraphics[width=0.3\textwidth]{UNET/Host_UI}
  }
  \qquad
  \subfloat[Client UI]{
    \includegraphics[width=0.3\linewidth]{UNET/Client_UI}
  }

  \caption{UNET default UI}
  \label{fig:unet_ui}
\end{figure}

\begin{figure}[p]
  \centering
  \includegraphics[width=\textwidth]{UNET/Main_Menu}
  \caption{Main Menu}
  \label{fig:main}
\end{figure}


\begin{figure}[p]
  \centering

  \subfloat[Host game screen]{
    \includegraphics[width=0.45\textwidth]{UNET/Host}
  }
  \qquad
  \subfloat[Client game screen]{
    \includegraphics[width=0.45\linewidth]{UNET/Client}
  }

  \caption{Main game loop after connection is established}
  \label{fig:in_game}
\end{figure}

\newpage

\subsubsection{PUN (Photon Unity Network)}
PUN is an example of a 3rd party solution for networking in Unity. The Photon Engine is a networking platfrom that allows developers to rent server space for multiplayer games. They also provide APIs for many different platforms and services such as matchmaking or chat. The implementation of this library was just as easy as with UNET


\section{Related work summary}
TODO: what I am going to focus on and why and this leads nicely into chapter 3
