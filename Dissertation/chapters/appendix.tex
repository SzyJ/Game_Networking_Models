\chapter{Basic concepts and terminolory}\label{app:terms}
\textbf{Client}: Within the context of this document, a client can be defined as a piece of software responsible to running the game (i.e. game client). A client can also be defined as a computer interacting with a server, however it is possible to run two different instances of a game client on a single machine. This can also be described as a networking endpoint.
\\
\\
\textbf{Online Multiplayer Game}: A video game can be defined as a simulation of a certain scenario that can be manipulated by the player of the game. When talking about online multiplayer games, it can be thought of as a simulation that runs on several clients connected by a network (e.g. LAN or the Internet) that is to be synchronised. When one player performs an action that effects the state of the simulation, this action should also be seen by all participants of this particular simulation instance and therefore the simulation should remain in the same state across all participating clients. Due to the nature of this paper, when discussing games or simulations that use the network, it can be assumed that the given instance would be played by N > 1 participants.
\\
\\
\textbf{Ping}: In network connections, the ping between different clients refers to the shortest amount of time that is needed for one client to send information to another and receive a response from this client. One client sends a ``ICMP echo request'' to another networked client (e.g. a game server). The receiving client, then responds with an ``ICMP echo reply'' back to the original device. The time between sending the request and receiving the reply, is the ping between the two clients.
\\
\\
\textbf{Lag}: The grater the ping between two connected clients, the bigger the difference in the state of each clients' simulation once an action to be synchronised is performed. When a change is made by one client, this change should be seen by other clients participating in the same simulation and lag occurs when this change does not appear instantaneous to the user.
\\
\\
\textbf{Jitter}: The difference in frequency that the messages are sent from a sender and received by the receiver. If a sending client sends packets at a constant rate, they would ideally arrive at the receiver's client at the same rate. This is not always the case however and could lead to some unwanted results in certain applications such as VOIP.


\chapter{Networking Model Attributes}\label{app:attributes}

\section{Central Server Model}
\begin{figure}[h!]
  \begin{tabular}{ c p{0.94\textwidth} }
    \faCheckCircle & Hardware is likely to be powerful enough to handle the stress of many simulations running simultaneously. \\
    \faCheckCircle & High bandwidth connection is likely to be used. Minimises the chance of high latency and network issues. \\
    \faCheckCircle & Ping to the server is likely to be similar for all players making the game more fair.  \\
    \faCheckCircle & No client can see the address of any other client. \\
    \faCheckCircle & The developer has a lot of control over what is allowed within the game. This allows for anti-cheating systems that are hard to bypass. \\
    \faTimesCircle & Expensive to rent out or buy and maintain server space. Letting players rent out servers makes the game more expensive for them. \\
    \faTimesCircle & Many servers have to be spread out evenly throughout the world to allow for low latency connections.  \\
    \faTimesCircle & People living in remote locations may not have low latency access to official servers.
  \end{tabular}
  \caption{The attributes of the central server model}
\end{figure}

\newpage

\section{Client Hosted Model}
\begin{figure}[!h]
  \begin{tabular}{ c p{0.94\textwidth} }
    \faCheckCircle & There are no additional costs to the publisher/developer when releasing the game. \\
    \faCheckCircle & Players in remote locations can play together with low latency. \\
    \faCheckCircle & The codebase is relatively easily transferable to a central server model if the need arises. \\
    \  & \  \\
    \faTimesCircle & The host player has neglegible ping to the server. This could be a large advantage.  \\
    \faTimesCircle & The host player is likely to use a consumer grade connection increasing the risk of packet loss and high latency \\
    \faTimesCircle & The host player could be using WiFi to host the game which could significantly increase the chance of packet loss. \\
    \faTimesCircle & The host's hardware may not be powerful enough to calculate each simulation step within an accptable tick rate. \\
    \faTimesCircle & If the host player is disconnected mid-game, a host migration will have to take place pausing the game for a few seconds or causing the game to finish unexpectadly. \\
    \faTimesCircle & The host player can see the IP address of each other player that they are playing with. \\
    \faTimesCircle & Cheating could be easy if the host player sends malicious packets to the clients pretending to be the game server.

  \end{tabular}
  \caption{The attributes of the client hosted model}
  \label{fig:ch_attributes}
\end{figure}

\newpage

\section{Peer to Peer model}
\begin{figure}[h!]
  \begin{tabular}{ c p{0.5\textwidth} }
    \faCheckCircle & Test good \\
    \faMinusCircle & Test attribute \\
    \faTimesCircle & Test bad
  \end{tabular}
  \caption{The attributes of the peer to peer model}
\end{figure}


\chapter{Jitter Variance table}

\begin{table}[!h]
  \centering
  \begin{tabular}{ l r r r }
    \toprule
    Delay between messages (ms) & WiFi & Ethernet & Localhost \\
    \midrule

     8     & 3.67E-06       & 3.74E-06 & 3.46E-06 \\
     64    & 8.81E-07       & 5.67E-07 & 4.61E-07 \\
     125   & 1.54E-06       & 8.68E-07 & 8.16E-07 \\
     250   & 1.46E-06       & 9.19E-07 & 7.72E-07 \\
     500   & 3.62E-05       & 1.63E-07 & 8.40E-08 \\
     1000  & 0.009563789805 & 3.18E-07 & 1.84E-07 \\

    \bottomrule
  \end{tabular}
  \caption{Table showing the variance values of the jitter seen in test result}
  \label{table:var_table}
\end{table}


\chapter{Developement Issues}
\begin{figure}[!h]
  \centering
  \includegraphics[width=\textwidth]{GNAT/sending_too_fast.png}
  \caption{Messages being broadcast with no delay}
  \label{fig:broadcast_too_fast}
\end{figure}

\begin{figure}[!h]
  \centering
  \includegraphics[width=\textwidth]{GNAT/sending_with_delay.png}
  \caption{Messages being broadcast with delay}
  \label{fig:broadcast_with_delay}
\end{figure}

\chapter{GNAT Code Snippets}

\section{Pre-compiled Headers}
When developing a project that uses code that will not change, such as the std library and winsock, it is inefficient to compile the code for these libraries every time the project code is compiled. To greatly improve compile times, precompiled headers are used.

\textbf{pch.h}
\begin{lstlisting}
// General
#include <string>
#include <iostream>
#include <thread>

// Data Structures
#include <map>
#include <vector>

// WinSock
#include <winsock2.h>
#include <Ws2tcpip.h>
#include <windows.h>

#pragma comment (lib, "ws2_32.lib")

// Custom
#include "log.h"
\end{lstlisting}

\textbf{pch.cpp}
\begin{lstlisting}
#include "pch.h"
\end{lstlisting}


\section{Logging library wrapper}
For the logging library, a 3rd party project was used called spdlog. The wrapper is used to use this library in predefined ways.


\textbf{Log.h}
\begin{lstlisting}
#pragma once
#include <memory>
#include <spdlog/spdlog.h>

namespace GNAT {
	class GNAT_Log {
	public:
		static void init();
		static void init_client();
		static void init_server();
		static void init_peer();
		static void init_connection();

		inline static std::shared_ptr<spdlog::logger>& getConnectionLogger() { return connection_logger; }
		inline static std::shared_ptr<spdlog::logger>& getServerLogger() { return server_logger; }
		inline static std::shared_ptr<spdlog::logger>& getPeerLogger() { return peer_logger; }
		inline static std::shared_ptr<spdlog::logger>& getClientLogger() { return client_logger; }

	private:
		const static int LOG_FILE_SIZE_IN_MB = 5;
		const static int ROTATING_FILE_COUNT = 3;

		static std::shared_ptr<spdlog::logger> connection_logger;
		static std::shared_ptr<spdlog::logger> server_logger;
		static std::shared_ptr<spdlog::logger> peer_logger;
		static std::shared_ptr<spdlog::logger> client_logger;
	};
}

// Loging Macros
#define CONNECT_LOG_FATAL(...) GNAT::GNAT_Log::getConnectionLogger()->fatal(__VA_ARGS__); std::cout << "  " << __VA_ARGS__ << std::endl
#define CONNECT_LOG_ERROR(...) GNAT::GNAT_Log::getConnectionLogger()->error(__VA_ARGS__); std::cout << "  " << __VA_ARGS__ << std::endl
#define CONNECT_LOG_WARN(...) GNAT::GNAT_Log::getConnectionLogger()->warn(__VA_ARGS__);	 std::cout << "  " << __VA_ARGS__ << std::endl
#define CONNECT_LOG_INFO(...) GNAT::GNAT_Log::getConnectionLogger()->info(__VA_ARGS__);	 std::cout << "  " << __VA_ARGS__ << std::endl
#define CONNECT_LOG_TRACE(...) GNAT::GNAT_Log::getConnectionLogger()->trace(__VA_ARGS__); std::cout << "  " << __VA_ARGS__ << std::endl

#define SERVER_LOG_FATAL(...) GNAT::GNAT_Log::getServerLogger()->fatal(__VA_ARGS__); std::cout << "  " << __VA_ARGS__ << std::endl
#define SERVER_LOG_ERROR(...) GNAT::GNAT_Log::getServerLogger()->error(__VA_ARGS__); std::cout << "  " << __VA_ARGS__ << std::endl
#define SERVER_LOG_WARN(...) GNAT::GNAT_Log::getServerLogger()->warn(__VA_ARGS__);	 std::cout << "  " << __VA_ARGS__ << std::endl
#define SERVER_LOG_INFO(...) GNAT::GNAT_Log::getServerLogger()->info(__VA_ARGS__);	 std::cout << "  " << __VA_ARGS__ << std::endl
#define SERVER_LOG_TRACE(...) GNAT::GNAT_Log::getServerLogger()->trace(__VA_ARGS__); std::cout << "  " << __VA_ARGS__ << std::endl

#define PEER_LOG_FATAL(...) GNAT::GNAT_Log::getPeerLogger()->fatal(__VA_ARGS__);	 std::cout << "  " << __VA_ARGS__ << std::endl
#define PEER_LOG_ERROR(...) GNAT::GNAT_Log::getPeerLogger()->error(__VA_ARGS__);	 std::cout << "  " << __VA_ARGS__ << std::endl
#define PEER_LOG_WARN(...) GNAT::GNAT_Log::getPeerLogger()->warn(__VA_ARGS__);		 std::cout << "  " << __VA_ARGS__ << std::endl
#define PEER_LOG_INFO(...) GNAT::GNAT_Log::getPeerLogger()->info(__VA_ARGS__);		 std::cout << "  " << __VA_ARGS__ << std::endl
#define PEER_LOG_TRACE(...) GNAT::GNAT_Log::getPeerLogger()->trace(__VA_ARGS__);	 std::cout << "  " << __VA_ARGS__ << std::endl

#define CLIENT_LOG_FATAL(...) GNAT::GNAT_Log::getClientLogger()->fatal(__VA_ARGS__); std::cout << "  " << __VA_ARGS__ << std::endl
#define CLIENT_LOG_ERROR(...) GNAT::GNAT_Log::getClientLogger()->error(__VA_ARGS__); std::cout << "  " << __VA_ARGS__ << std::endl
#define CLIENT_LOG_WARN(...) GNAT::GNAT_Log::getClientLogger()->warn(__VA_ARGS__);	 std::cout << "  " << __VA_ARGS__ << std::endl
#define CLIENT_LOG_INFO(...) GNAT::GNAT_Log::getClientLogger()->info(__VA_ARGS__);	 std::cout << "  " << __VA_ARGS__ << std::endl
#define CLIENT_LOG_TRACE(...) GNAT::GNAT_Log::getClientLogger()->trace(__VA_ARGS__); std::cout << "  " << __VA_ARGS__ << std::endl
\end{lstlisting}

\textbf{Log.cpp}

\begin{lstlisting}
#include "pch.h"
#include "log.h"
#include <spdlog/sinks/rotating_file_sink.h>
#include <ctime>

namespace GNAT {
	std::shared_ptr<spdlog::logger> GNAT_Log::connection_logger;
	std::shared_ptr<spdlog::logger> GNAT_Log::server_logger;
	std::shared_ptr<spdlog::logger> GNAT_Log::peer_logger;
	std::shared_ptr<spdlog::logger> GNAT_Log::client_logger;

	void GNAT_Log::init() {
		spdlog::set_pattern("%^[%T] %n: %v%$");

		try
		{
			peer_logger = spdlog::rotating_logger_mt("CONN", "Logs\\CONN-" + std::to_string(std::time(0)) + ".log", 1024 * 1024 * LOG_FILE_SIZE_IN_MB, ROTATING_FILE_COUNT);
			server_logger = spdlog::rotating_logger_mt("SERV", "Logs\\SERV-" + std::to_string(std::time(0)) + ".log", 1024 * 1024 * LOG_FILE_SIZE_IN_MB, ROTATING_FILE_COUNT);
			peer_logger = spdlog::rotating_logger_mt("PEER", "Logs\\PEER-" + std::to_string(std::time(0)) + ".log", 1024 * 1024 * LOG_FILE_SIZE_IN_MB, ROTATING_FILE_COUNT);
			client_logger = spdlog::rotating_logger_mt("CLNT", "Logs\\CLNT-" + std::to_string(std::time(0)) + ".log", 1024 * 1024 * LOG_FILE_SIZE_IN_MB, ROTATING_FILE_COUNT);
		}
		catch (const spdlog::spdlog_ex& ex)
		{
			std::cout << "Log initialization failed: " << ex.what() << std::endl;
		}
	}

	void GNAT_Log::init_client() {
		spdlog::set_pattern("%^[%T] %n: %v%$");

		try
		{
			client_logger = spdlog::rotating_logger_mt("CLNT", "Logs\\CLNT-" + std::to_string(std::time(0)) + ".log", 1024 * 1024 * LOG_FILE_SIZE_IN_MB, ROTATING_FILE_COUNT);
		}
		catch (const spdlog::spdlog_ex& ex)
		{
			std::cout << "Log initialization failed: " << ex.what() << std::endl;
		}
	}

	void GNAT_Log::init_server() {
		spdlog::set_pattern("%^[%T] %n: %v%$");

		try
		{
			server_logger = spdlog::rotating_logger_mt("SERV", "Logs\\SERV-" + std::to_string(std::time(0)) + ".log", 1024 * 1024 * LOG_FILE_SIZE_IN_MB, ROTATING_FILE_COUNT);
		}
		catch (const spdlog::spdlog_ex& ex)
		{
			std::cout << "Log initialization failed: " << ex.what() << std::endl;
		}
	}

	void GNAT_Log::init_peer() {
		spdlog::set_pattern("%^[%T] %n: %v%$");

		try
		{
			peer_logger = spdlog::rotating_logger_mt("PEER", "Logs\\PEER-" + std::to_string(std::time(0)) + ".log", 1024 * 1024 * LOG_FILE_SIZE_IN_MB, ROTATING_FILE_COUNT);
		}
		catch (const spdlog::spdlog_ex& ex)
		{
			std::cout << "Log initialization failed: " << ex.what() << std::endl;
		}
	}

	void GNAT_Log::init_connection() {
		spdlog::set_pattern("%^[%T] %n: %v%$");

		try
		{
			connection_logger = spdlog::rotating_logger_mt("CONN", "Logs\\CONN-" + std::to_string(std::time(0)) + ".log", 1024 * 1024 * LOG_FILE_SIZE_IN_MB, ROTATING_FILE_COUNT);
		}
		catch (const spdlog::spdlog_ex& ex)
		{
			std::cout << "Log initialization failed: " << ex.what() << std::endl;
		}
	}
}
\end{lstlisting}
