\begin{table}[!h]
  \centering
  \begin{tabular}{ l l p{0.3\textwidth} l }
    \toprule
    Message Type & Message Code & Description & Example Payload \\
    \midrule
    Join Request &
      \lstinline[]$JR$ &
      Allows a client to send a join request to the server. &
      \lstinline[]$JR$ \\
    \addlinespace[10pt]
    Join Acknowledgement &
      \lstinline[]$JA$ &
      Allows the server to confirm that the client's information has been saved. Is followed by 1 byte indicating the client's ID &
      \lstinline[]$JA1$ \\
    \addlinespace[10pt]
    Ping Request &
      \lstinline[]$PQ$ &
      Message instructing the recipiant to reply with \lstinline[]$PS$. Can be used to time the delay in this connection.&
      \lstinline[]$PQ$ \\
    \addlinespace[10pt]
    Ping Response &
      \lstinline[]$RS$ &
      This should be sent whenever a \lstinline[]$PQ$ message is received. &
      \lstinline[]$RS$ \\
    \addlinespace[10pt]
    Update &
      \lstinline[]$UP$ &
      Used by a client to update it's value on the server. Is followed by 1 byte representing the new value. &
      \lstinline[]$UP9$ \\
    \addlinespace[10pt]
    Define &
      \lstinline[]$DF$ &
      Used by the server to define the initial values for each of the clients connected to this instance. It is followed by a non-zero, even amount of bytes representing the client ID and it's value pair. &
      \lstinline[]$DF1020304050$ \\
    \addlinespace[10pt]
    Current State &
      \lstinline[]$CS$ &
      Used by the server to broadcast it's real state to all clients. When this is received, clients are expected to update their local state to this. It is followed by a non-zero, even amount of bytes representing the client ID and it's value pair. &
      \lstinline[]$CS1927344157$ \\

    \bottomrule
  \end{tabular}
  \caption{Table showing the message codes for distinguishing messages from each other and how each one is to be used}
  \label{table:message-codes}
\end{table}
