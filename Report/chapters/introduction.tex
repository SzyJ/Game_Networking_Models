\chapter{Introduction}
Throughout this project, I will be exploring the different methods of providing a synchronised, multiplayer gaming experiance on two or more computers on a network. I will be covering the networking basics of how two game clients running on seperate machines can communicate with each other as well as exploring the relevant details of how and why the networking systems in games are designed the way they are.


I will explore the existing methods of writing an online multiplayer game from scratch and aim to provide an implementation of a networking library for online games to communicate without a central server. I also aim to provide analysis of efficieny of different methods of achieving this goal.

\section{Basic concepts and Terminolory}
First of all, I will define some basic concepts and terminolory that will be used throughout this document.
\\
\\
\textbf{Client:} Within the context of this document, a client can be defined as a piece of software responsible to running the game (i.e. game client). A client can also be defined as a computer interacting with a server, however it is possible to run two different instances of a game client on a single machine.
\\
\\
\textbf{Online Multiplayer Game:} A video game can be defined as a simulation of a certain scenario that can be manipulated by the player of the game. When talking about online multiplayer games, it can be thought of as a simulation that runs on several clients connected by a network (e.g. LAN or the Internet) that is to be synchronised. When one player performs an action that effects the state of the simulation, this action should also be seen by all participants of this perticular simulation instance and therefore the simulation should remain in the same state across all participating clients.
\\
\\
\textbf{Ping:} In network connections, the ping between different clients refers to the shortest amount of time that is needed for one client to send information to another and receive a response from this client. One client sends a ``ICMP echo request'' to another networked client (e.g. a game server). The receiving client, then responds with an ``ICMP echo reply'' back to the original device. The time between sending the request and receiving the reply, is the ping between the two clients.
\\
\\
\textbf{Lag:} The grater the ping between two connected clients, the bigger the difference in the state of each clients' simulation once an action to be synchronised is performed. When a change is made by one client, this change should be seen by other clients participating in the same simulation and lag occurs when this change does not appear instentaneous to the user.

\section{Networking principles in games}
TODO: talk a bit about UDP, TCP/IP and routing...

\subsection{The Centralised Server Model}
Most online multiplayer games that are played today, make use of the ``centralised server model'' for sinchronising the simulation state between several clients participating in the same simulation. This means that in an example of a First Person Shooter (FPS), if one player presses the ``jump'' key, their character will jump and this information is would also be sent over to the game server. The server would then send the information that this player has jumped, to all other clients. There is a potential problem here however. Given that the ping between the server and client A is \(\alpha\) and between the server and client B is \(\beta\), the time between client A pressing an input and client B being notified of this input can not be less than $\alpha+\beta$ and due to the limitations of physics $\alpha>0$ and $\beta>0$. This means that at any given time the lag experianced between clients A and B will be more than $\alpha+\beta$ when processing times are factored in too. This leads to a problem of a poor experience for a client with a high ping to the server, as they will receive the updates from the server later than every other player and therefore be at a disadvantage if the game requires real time reactions. Unfortunately under some implementations, this also results in a poor experiance for every other player, who despite having resonable ping to the server, can be shot from behind cover by a laggy player who fired a shot before the cover was reached.
\subsubsection{Possible solutions to the variable ping problem}
TODO: talk about region locks, and disconnecting players with high ping but ping can be high even in normal conditions

\subsubsection{Battelfield 1\textsuperscript{\textregistered} implementation}
An interesting approach to this issue has been implemented by DICE in the game Battelfield 1. Given 2 players; player A with a low ping to the server and player B with a ping of <150ms to the server.

When player B fires at a moving player A, player B's client will perform the check concluding that player A has been hit and this information is sent to the server. The server will then perform it's own checks and if the server agrees that this hit is possible, then it sends the hit confirmation to player B and damage information to player A. This approach is called Clientside-Server Authoritative as while the hit registration is calculated on clientside, the server must still confirm that this is valid.

Concidering another scenario, suppose that player A still has a low ping to the server but player B, now has the ping of >150ms. An icon will appear on player B's UI showing an ``aim-lead'' indicator. Now when the shot is fired in the same scenario, the hit will not register anymore as the hit registration has switched from Clientside-Server Authoritative to Fully-Server Authoritative, meaning that the check is performed only once the shot information is received by the server.

Whilest this implementation makes the game feel less responsive for players with high ping, it provides a lot more fairness for everyone else and allow for players with different pings play in a more fair way.
